% Options for packages loaded elsewhere
\PassOptionsToPackage{unicode}{hyperref}
\PassOptionsToPackage{hyphens}{url}
%
\documentclass[
]{article}
\usepackage{amsmath,amssymb}
\usepackage{lmodern}
\usepackage{iftex}
\ifPDFTeX
  \usepackage[T1]{fontenc}
  \usepackage[utf8]{inputenc}
  \usepackage{textcomp} % provide euro and other symbols
\else % if luatex or xetex
  \usepackage{unicode-math}
  \defaultfontfeatures{Scale=MatchLowercase}
  \defaultfontfeatures[\rmfamily]{Ligatures=TeX,Scale=1}
\fi
% Use upquote if available, for straight quotes in verbatim environments
\IfFileExists{upquote.sty}{\usepackage{upquote}}{}
\IfFileExists{microtype.sty}{% use microtype if available
  \usepackage[]{microtype}
  \UseMicrotypeSet[protrusion]{basicmath} % disable protrusion for tt fonts
}{}
\makeatletter
\@ifundefined{KOMAClassName}{% if non-KOMA class
  \IfFileExists{parskip.sty}{%
    \usepackage{parskip}
  }{% else
    \setlength{\parindent}{0pt}
    \setlength{\parskip}{6pt plus 2pt minus 1pt}}
}{% if KOMA class
  \KOMAoptions{parskip=half}}
\makeatother
\usepackage{xcolor}
\IfFileExists{xurl.sty}{\usepackage{xurl}}{} % add URL line breaks if available
\IfFileExists{bookmark.sty}{\usepackage{bookmark}}{\usepackage{hyperref}}
\hypersetup{
  hidelinks,
  pdfcreator={LaTeX via pandoc}}
\urlstyle{same} % disable monospaced font for URLs
\setlength{\emergencystretch}{3em} % prevent overfull lines
\providecommand{\tightlist}{%
  \setlength{\itemsep}{0pt}\setlength{\parskip}{0pt}}
\setcounter{secnumdepth}{-\maxdimen} % remove section numbering
\ifLuaTeX
  \usepackage{selnolig}  % disable illegal ligatures
\fi

\author{}
\date{}

\begin{document}

\hypertarget{gpt-aa-gpt-3-academic-assistant}{%
\section{GPT-AA (GPT-3 Academic
Assistant)}\label{gpt-aa-gpt-3-academic-assistant}}

GPT-AA is a set of tools for interacting with the GPT-3 API for the
purpose of human-free open question grading. GPT-AA provides a
constraint \& abstraction layer for interacting with GPT-3 as a
classifier, as opposed to its classical application as a purely
generative AI.

\hypertarget{core-concepts}{%
\section{Core Concepts}\label{core-concepts}}

\begin{enumerate}
\def\labelenumi{\arabic{enumi}.}
\tightlist
\item
  Examples
\end{enumerate}

\begin{itemize}
\item
  ``Given any text prompt, the API will return a text completion,
  attempting to match the pattern you gave it. You can ``program'' it by
  showing it just a few examples of what you'd like it to do; its
  success generally varies depending on how complex the task is. The API
  also allows you to hone performance on specific tasks by training on a
  dataset (small or large) of examples you provide, or by learning from
  human feedback provided by users or labelers."
\item
  GPT-3 is a general AI, meaning that it is capable of a wide variety of
  tasks. In our case, we are looking to use GPT-3 as a classification
  algorithm.
\item
  Using GPT-3 is only permitted through integration with its API. Since
  GPT-3 can parse english, feeding it a handful of examples is
  sufficient information for GPT-3 to infer deep levels of context and
  meaning from both the labels and the examples.
\item
  This means, for example, that GPT-3 can interpolate between scores
  i.e.~GPT-3 could be fed examples of a 0/10, 5/10, and a 10/10 and be
  able to infer what a 3/10 or 7/10 would look like.
\end{itemize}

\begin{enumerate}
\def\labelenumi{\arabic{enumi}.}
\setcounter{enumi}{1}
\tightlist
\item
  The Config File
\end{enumerate}

\begin{itemize}
\tightlist
\item
  genConfig.py is a convenience tool for inputting these training
  examples into the required format for classification. Essentially,
  genConfig.py will ask for a series of answers at specific scores, and
  then store them in an appropriately formated JSON file. Any (sensible)
  novel answer can be scored against a generated configuration file.
\end{itemize}

\begin{enumerate}
\def\labelenumi{\arabic{enumi}.}
\setcounter{enumi}{2}
\tightlist
\item
  GPT-3 Remote File Upload
\end{enumerate}

\begin{itemize}
\item
  In order to maximize accuracy, GPT-3 requires the examples to be
  stored remotely on the OpenAI server. The file outputted by
  genConfig.py is formatted in line with OpenAI's standard.
\item
  uploadFile.py is a convenience tool for rapidly uploading the output
  of genConfig.py (by default saved to ``gpt3Config.json''). After
  uploading, the program will return the metadata necessary to point to
  the remotely hosted file. The ``id'' field of the metadata output will
  ultimately be used as an input into the querying program.
\end{itemize}

\begin{enumerate}
\def\labelenumi{\arabic{enumi}.}
\setcounter{enumi}{3}
\tightlist
\item
  Queries
\end{enumerate}

\begin{itemize}
\item
  After uploading a configuration file and saving its corresponding id,
  the main program is ready to be used. Index.py is the entry point for
  the primary functionality. Index.py accepts 2 primary command line
  arguments:

  \begin{itemize}
  \item
    -f=``fileId'' or --file=``fileId'' This argument provides a file ID
    to be referenced by the classification function, and corresponds the
    the output of uploadFile.py. e.g.~\textgreater\textgreater{} python
    index.py --file=``file-ksfjdskajfkjKXJKF''
  \item
    -q=``Answer to be graded'' or --query=``Answer to be graded'' This
    argument provides an answer to be graded against the configuration
    file's examples. e.g.~\textgreater\textgreater{} python index.py
    --query="This is th
  \end{itemize}
\end{itemize}

\end{document}